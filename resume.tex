% LaTeX resume using res.cls
\documentclass{res}
%\usepackage{helvetica} % uses helvetica postscript font (download helvetica.sty)
%\usepackage{newcent}   % uses new century schoolbook postscript font  
\setlength{\topmargin}{-0.6in}  % Start text higher on the page 
\setlength{\textheight}{9.8in}  % increase textheight to fit more on a page
\setlength{\headsep}{0.2in}     % space between header and text
\setlength{\headheight}{12pt}   % make room for header
\usepackage{fancyhdr}  % use fancyhdr package to get 2-line header
\usepackage{hyperref}
\renewcommand{\headrulewidth}{0pt} % suppress line drawn by default by fancyhdr
\lhead{\hspace*{-\sectionwidth}Gregory ``Rory'' Ksionda} % force lhead all the way left
\rhead{Page \thepage}  % put page number at right
\cfoot{}  % the footer is empty
\pagestyle{fancy} % set pagestyle for the document

\begin{document} 
\thispagestyle{empty} % this page does not have a header
\name{Gregory ``Rory'' Ksionda}
\address{ksiondag846@gmail.com\\
(520) 329-5081\\
\url{https://github.com/ksiondag}\\
Seattle, WA}


\begin{resume}
   
\section{Employment} 
\vspace{-0.1in} 
  \begin{tabbing}
    \hspace{2.2in}\= \hspace{2.2in}\= \kill
    {\bf Fullstack Engineer} \> {\bf Mixpanel, Inc.}     \>August 2017-Present
  \end{tabbing}\vspace{-5pt}

  \vspace{-20pt}\begin{tabbing}
    \hspace{2.2in}\= \hspace{2.2in}\= \kill
    {\bf Software Engineer} \>{\bf Rincon Research Corp.} \> May 2012-Feburary 2017
  \end{tabbing}\vspace{-5pt}
  \vspace{-20pt}\begin{tabbing}
    \hspace{2.2in}\= \hspace{2.2in}\= \kill
    {\bf Casual Employee (Intern)} \>{\bf Rincon Research Corp.} \> Summer 2011-May 2012
  \end{tabbing}\vspace{-5pt}

\section{Professional Projects}
  {\bf Lexicon (Mixpanel, June 2018-Present)}: Actively upgrading Mixpanel's implementation doucmentation tool
  in the Mixpanel product. Developing an internal REST API in Python with Django, and developing an UI with
  TypeScript and WebComponents.

  {\bf RBAC (Rincon, 2016)}: Ugraded an internal documentation-access tool using an RBAC-sytem made in Django.
  Fixed a broken testing implementation and created test for each updated bit of code. The tool
  verified: project-access, up-to-date training, and other valid role prerequisites. 

  {\bf Code-Health (Rincon, 2015)}: Developed  testing infrastructure in Python which would allow for 
  known-failing tests. The idea was to decouple tests from bug fixes, and measure error-prone code.

  {\bf USRP (Rincon, 2013)}: Developed the software for a signal analysis experiment that synced radio collection across
  an arbitrary number of USRPs (Universal Software Defined Radios). General collection loop created in Python,
  with some very specific timing code made in C++.

  {\bf Telescope (Rincon, Summer 2011)}: Internship with a fellow computer engineer, a mechanical engineer, and a
  physicist, created software in Python to interact with a motorized telescope to track
  and take pictures of the ISS with the aid of TLEs and an error-correcting UI made in Qt4.
\vspace{0.1in}
    
 
\section{Hobbyist Projects}
All project code available on GitHub account: \url{https://github.com/ksiondag}
  \begin{itemize}
    \item \href{https://github.com/ksiondag/base_webapp}{base\_webapp}: a Django-backend, React-frontend
    base for various project ideas.
    \item \href{https://github.com/ksiondag/pygame_skeleton}{pygame\_skeleton}: a basic skeleton to dive
    into a pygame project (made to help a friend get into programming).
    \item \href{https://github.com/ksiondag/browsing_costs}{browsing\_costs}: first React attempt, a
    Chrome extension that gamifies site-blocking.
    \item \href{https://github.com/ksiondag/gamejam_clone}{gamejam\_clone}: a point and click platformer,
    browser-based game.
    \item \href{https://github.com/ksiondag/gamejam_train}{gamejam\_train}: a text adventure time-travel
    game written in node.js, using git as the mechanics of time-travel.
  \end{itemize}
\vspace{0.1in}

  
\section{Languages} 
\vspace{0.1in}
  {\bf Python (2.7 and 3.6),} 5+ years of experience\\
  {\bf JavaScript/TypeScript,} 3 years of experience\\
  {\bf PostgreSQL,} 1 year of experience\\
  {\bf C++,} 1 year of experience
  
\section{Education}
\vspace{0.1in} 
  University of Arizona, Tucson, AZ  \\        
  Bachelor of Science, Computer Engineering, May 2012   \\       
  G.P.A. 4.0/4.0          
    
 
\end{resume}
\end{document}